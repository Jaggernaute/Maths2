\documentclass{config/homework}
\usepackage[utf8]{inputenc}
\usepackage[english]{babel}
\usepackage{helvet}

\title{\fontfamily{phv}\selectfont MAT2. Algebre lineaire.}
\author{Videcoq Lucas \\ ISTIC - Universite de Rennes 1, \textit{\thedate}}
\date{Hiver 2023}
\version{1.0}


\begin{document}


    \begin{titlepage}

        \maketitle

        \begin{abstract}
            Ce document traitera de l'exercice n°42 du cours de la feuille correspondant
            au chapitre 1 du cour de MAT2 sur l'Algebre lineaire. \\
            Cet exercice a faire a la maison est a rendre pour le 07/02/2023.\\
            Ce document est entierement redige en LaTeX. \\
            Le code source est disponible a l'adresse suivante: \url{https://github.com/Jaggernaute/Maths2}
        \end{abstract}
        \vspace{5em}
        \center\includegraphics{images/univ-logo}

        \newpage

        \tableofcontents

        \vspace{5em}

        \begin{contributors}
            \vspace{.25em}
            Redaction : \textbf{Lucas Videcoq} \\
            Cours dispense par : \textbf{Prof. Felix Ulmer} \\
            Bibliographie : \textbf{Otto Bretscher}, Linear Algebra with applications,
            3rd edition \\
            Travaux diriges : \textbf{Prof. Anne Virrion} \\
            \vspace{.25em}
        \end{contributors}

    \end{titlepage}

    \section{Ex: 42 - Les systèmes dits homogènes}\label{sec:section}

    \subsection{a - Les systèmes homogènes sont consistants}\label{subsec:a}
    Sont touts les systèmes homogènes consistants? \\

    Un systeme est dit inconsistent si sa matrice est de rang inferieur a son nombre de lignes, ou
    si sa formne reduite echelonnee est de la forme:
    \[
        \left[
            \begin{array}{rrrr|r}
                1 & 0 & \cdots & 0 & b_{1} \\
                0 & 0 & \cdots & 0 & b_{2} \\
                \vdots & \vdots & \ddots & \vdots & \vdots \\
                0 & 0 & \cdots & 0 & b_{n}
            \end{array}
            \right]
    \]
    avec $b_i \in \mathbb{R}$ et $b_i \neq 0$ pour tout $i \in \{1, \cdots, n\}$.\\
    Ici nous sommes face a un systeme homogene de la forme $A\vec{x}=\vec{0}$ , donc nous avons $b_i = 0$
    pour tout $i \in \{1, \cdots, n\}$, et donc le systeme est consistent, puisque sa matrice est de rang 1,
    et sa forme reduite echelonnee est de la forme:
    \[
        \left[
            \begin{array}{rrrrr|r}
                1 & 0 & \cdots & 0 & 0 & 0 \\
                0 & 0 & \cdots & 0 & 0 & 0 \\
                \vdots & \vdots & \ddots & \vdots & \vdots & \vdots \\
                0 & 0 & \cdots & 0 & 0 & 0
            \end{array}
        \right]
    \]
    \noindent\fbox{%
        Donc tous les systemes homogenes sont consistants.
    }
    \\ \\
    Exemple:\\
    Soit le systeme homogene suivant:
    \[
        \begin{cases}
            x_1 + 2x_2 + 3x_3 = 0 \\
            2x_1 + 4x_2 + 6x_3 = 0 \\
            3x_1 + 6x_2 + 9x_3 = 0
        \end{cases}
    \]
    \[
        \left[
            \begin{array}{rrr|r}
                1 & 2 & 3 & 0 \\
                2 & 4 & 6 & 0 \\
                3 & 6 & 9 & 0
            \end{array}
        \right]
    \]

    En utilisant l'algorithme de Gauss et en eliminant la premiere colonne on obtient la matrice echelonnee reduite suivante:

    \[
        \left[
        \begin{array}{rrr|r}
                1 & 2 & 3 & 0 \\
                0 & 0 & 0 & 0 \\
                0 & 0 & 0 & 0
        \end{array}
        \right]
    \]

    \noindent\fbox{%
        On peu en deduire ce systeme est donc consistent.
    }


    \newpage

    \subsection{b - Les systèmes homogènes avec moins d’équations que d’inconnues.}\label{subsec:b}
    Un systeme homogene avec moins d'equations que d'inconnues possede t'il une infinite solution? \\

    Un systeme homogene avec strictement moins d'equations que d'inconnues possede une infinite une infinité de
    n-uplets solutions, car il est possible de trouver un vecteur
    $\vec{x}$ tel que $A\vec{x} = \vec{0}$, et donc $\vec{x}$
    est une solution du systeme.\\

    Exemple:\\
    Soit le systeme homogene suivant:
    \[
        \begin{cases}
            x_1 + 2x_2 + 3x_3 = 0 \\
            2x_1 + 4x_2 + 6x_3 = 0
        \end{cases}
    \]
    \[
        \left[
            \begin{array}{rrr|r}
                1 & 2 & 3 & 0 \\
                2 & 4 & 6 & 0
            \end{array}
        \right]
    \]

    En utilisant l'algorithme de Gauss et en eliminant la premiere colonne
    on obtient la matrice echelonnee reduite suivante:

\[
        \left[
        \begin{array}{rrr|r}
                1 & 2 & 3 & 0 \\
                0 & 0 & 0 & 0
        \end{array}
        \right]
    \]

    \noindent\fbox{%
        On peu en deduire ce systeme est donc consistent.
    }
    \newpage


    \subsection{c - Solutions d'un système homogène de la forme $\vec{x_1} + \vec{x_2}$}\label{subsec:c}

    \newpage


    \subsection{d - Solutions d'un système homogène de la forme $k\vec{x}$}\label{subsec:d}

    \newpage



\end{document}
